\documentclass[10pt,pdf,hyperref={unicode}]{beamer}


% \documentclass[aspectratio=43]{beamer}
% \documentclass[aspectratio=1610]{beamer}
% \documentclass[aspectratio=169]{beamer}

\usepackage{multicol}
\usepackage{lmodern}
\usepackage{lipsum}
\usepackage{marvosym}

% подключаем кириллицу 
%\usepackage[T1,T2A]{fontenc}
\usepackage[lutf8]{luainputenc}
\usepackage[english,russian]{babel}

% отключить клавиши навигации
\setbeamertemplate{navigation symbols}{}

% тема оформления
\usetheme{CambridgeUS}
% цветовая схема
\usecolortheme{crane}

\usepackage{fontspec}
        \defaultfontfeatures{Ligatures={TeX}}
        \setmainfont{Ubuntu}
        \setsansfont{Ubuntu}
        \setmonofont{Ubuntu Mono}
    \usepackage[english,russian]{babel}

\date{}

\title[]{...ers,Developers,Developers,Developers,Dev...}   
%\subtitle{Use beamer everywhere you are}
\author[]{Кирпиченков Денис} 
\institute[]{Naumen}
%\date{\today} 
% \logo{\includegraphics[height=5mm]{images/logo.png}\vspace{-7pt}}

\begin{document}

% титульный слайд
\begin{frame}
\titlepage
\end{frame} 

\begin{frame}
\frametitle{История о разработке одного продукта} 

\begin{itemize}
\item Время разработки 10 лет
\item 3 перерождения: python, java, brand new java с GWT
\item 3 000 000 строк java-кода, 125 000 строк XML файлов,  60 000 строк скриптов
\item Комманда из 20 разработчиков
\end{itemize}

\end{frame}

\begin{frame}
\frametitle{Read,Code,Debug,Test... Repeat} 
\end{frame}

\begin{frame}
\frametitle{Что общего у (почти) всех программистов} 

\begin{columns}
\column{0.5\textwidth}
%\includegraphics[width=1.1\textwidth]{./vostorg.png}
%
\includegraphics[width=1.1\textwidth]{./vostorg2.png}
\column{0.5\textwidth}
\includegraphics[width=1.1\textwidth]{./child01.png}

\includegraphics[width=1.1\textwidth]{./child02.png}
\end{columns}

\end{frame}

\begin{frame}
\frametitle{void function career() \{ while(1) \{ doCodeALot() \} \} }
\framesubtitle{ обощенный путь разработчика }
\end{frame}

\begin{frame}
\frametitle{ Embedded решения }
\end{frame}


\begin{frame}
\frametitle{Enterprise разработка или сайтик для души}
\end{frame}

\begin{frame}
\frametitle{ Mobile apps }
\end{frame}

\begin{frame}
\frametitle{ Главные знания }
\end{frame}

\begin{frame}
\frametitle{ \textasciitilde Главные знания }
\framesubtitle{ что не стоит знать }
\end{frame}

\begin{frame}
\frametitle{ Finale }
\end{frame}


\end{document}